\documentclass[12pt,letterpaper,onecolumn,oneside]{article}
\usepackage[margin=3cm]{geometry}
\usepackage{graphicx}
\usepackage[colorlinks=true,linkcolor=blue]{hyperref}
\usepackage[page]{appendix}
\usepackage{amsfonts}
\usepackage{amsmath}
\usepackage{amssymb}
\usepackage{bbold}
\usepackage{filecontents}
\usepackage[round,nonamebreak]{natbib}
\usepackage{mathtools}

%% I am embedding the bibliography in the latex file for simplicity !!

\begin{filecontents}{seismic.bib}
@misc{iaspei2011,
author ={{I}nternational {S}eismological {C}entre},
title = {{IASPEI} Standard Phase List}, 
url = {http://www.isc.ac.uk/standards/phases/},
publisher = {Internatl. Seis. Cent.}, 
address = {Thatcham, United Kingdom},
year={2011},
howpublished = {\url{http://www.isc.ac.uk/standards/phases/}}
}

@article{Arora2013,
  author = {Nimar S. Arora and Stuart Russell and Erik Sudderth},
  year = {2013},
  month = {April},
  title = {{NET-VISA}: Network Processing Vertically Integrated Seismic 
    Analysis},
  journal = {Bulletin of the Seismological Society of America},
  volume = {103 no. 2A},
  pages = {709--729}
}

}
\end{filecontents}

\begin{document}

\title{Seismic 2-D}
\author{
\textit{Nimar S. Arora}\\
Bayesian Logic, Inc.\\
Union City, CA 94587
\and
\textit{Stuart Russell}\\
Dept. of Computer Science\\
Berkeley, CA 94720
}
\maketitle

\begin{abstract}
The goal of this problem is to detect and localize seismic events on a
simulated two-dimensional world (the surface of a perfect sphere) given
signals collected over a fixed time interval at a number of seismic
stations. This is a simplification of the real problem
\citep{Arora2013}, and is designed as a challenge problem for
research in probabilistic programming languages.
\end{abstract}

\section{Generative Model}

\subsection{Events}

Events are generated by a space-time Poisson process over the surface of
the earth with rate parameter $\lambda_e$. This Poisson process is
assumed to be homogenous in both space and time. In other words, the
number of events in any time interval $T$ over the surface-area of the
earth, with radius $R$, has a Poisson distribution with rate $\lambda_e
4 \pi R^2 T
$. Further, it follows that the time of each event is uniformly
distributed in $[0,T]$, and the location is uniformly distributed over
the surface of the earth. If locations are represented by longitude and
latitude then one can equivalently state that longitudes are uniformly
distributed over $[-180, 180]$ and the $sin$ of the latitude is
uniformly distributed over $[-1, 1]$.

More formally, if $e$ is the set of events in the time interval of length
$T$ over the surface of the earth, then
\begin{align*}
  |e| & \sim \text{Poisson}(\ \cdot\ ;\ \lambda_e 4 \pi R^2 T), \\
\intertext{and for each event $e^i$ with time, longitude, and latitude
  given by $e^i_t$, $e^i_{l1}$, and $e^i_{l2}$ respectively,}
  e^i_t & \sim \text{Uniform}(\ \cdot\ ;\ 0, T) \\
  e^i_{l1} & \sim \text{Uniform}(\ \cdot\ ;\ -180, 180) \\
  \sin(e^i_{l2}) & \sim \text{Uniform}(\ \cdot\ ;\ -1, 1).
\intertext{The event magnitude, $e^i_m$ is distributed as per an
  exponential distribution with rate $\lambda_m$ = log(10), and a
  minimum value of $2$, i.e.,}
  e^i_m & \sim \text{Exponential}(\ \cdot\ ;\ \lambda_m, location=2)
\end{align*}

\subsection{True Detections}

The seismic energy from an event travels radially outwards in distinct
phases, each of which may or may not be detected by a station depending
on local noise levels. For example, the energy can travel via different
modes of transmission (compression waves, shear waves) and along
different layers of the earth, the so called {\em body} waves or along
the surface, refer to \citet{iaspei2011} for a full list. In this work we only
consider the first arriving phase, the {\em P} phase.

We define a true detection $\Lambda^{ik}$ as the moment of first arrival
of the energy from an event $i$ at a seismic station $k$. Various signal
processing algorithms are applied on the raw waveforms to detect an
arrival, and then station processing algorithms collect various
attributes of the
detection such as time, azimuth, slowness, and amplitude referenced by
$\Lambda^{ik}_t$, $\Lambda^{ik}_z$, $\Lambda^{ik}_s$, and
$\Lambda^{ik}_a$ respectively. Time is quite obviously the detection time
of the energy, azimuth refers to the geographical direction of the
incoming seismic waves, and amplitude is the height of the initial
peak. Slowness is a more peculiar term, it refers to the inverse of the
apparent surface speed of the waves, which will become clearer shortly.

\subsubsection{Detection Probability}

A number of the subsequent formulae depend on the great-circle distance
between two points on the surface of the earth. The formula for the
distance between two points at locations $a=(lon_1, lat_1)$ and 
$b=(lon_2, lat_2)$ is given by,
\[
dist(a,b) \\
= cos^{-1} ( sin(lat_1) * sin(lat_2) + cos(lat_1) * cos(lat_2) *
cos(lon_2 - lon_1) ) \, .
\]
In the following, we will denote the great-circle distance between the
location of event $i$, $e^i_l$ and station $k$, $s^k_l$ as
$\Delta_{ik}$, expressed in degrees. This distance lies in the interval
$[0,180]$.

The probability that an event $i$ is detected at station $k$ is a
function of the event magnitude and the great-circle distance between
the event and the station. We model this probability as a logistic
function,
\[\text{logistic}
(\mu_{d0}^k + \mu_{d1}^k e^i_m + \mu_{d2}^k \Delta_{ik}) \ .\]

\subsubsection{Detection Time}

The theoretical travel time of a seismic wave at a distance of $\delta$ is
given by the travel time function,
\[ I_T(\delta) = -.023 * \delta^2 + 10.7 * \delta\, + 5.\]

The detection time is a Laplacian centered near the theoretical detection time,
\[ \Lambda_t^{ik} \sim \text{Laplacian}(\ \cdot \ , e^i_t + I_T(\Delta_{ik}) +
\mu_t^k \  , \ \theta_t^k) . \]

\subsection{Detection Azimuth}

The azimuth of location $b=(lon_2, lat_2)$ as observed from location
$a=(lon_1, lat_1)$ is given by the function $G_z(a, b) \in [0,
  360)$. Where $0$ is due north and $180$ is due south.
\begin{align*}
G_z(a,b) &= \begin{dcases*}
  G'_z(a,b) & when $sin(lon_2 - lon_1) >= 0$ \\
  360 - G'_z(a,b) & otherwise
\end{dcases*}
\intertext{where,}
G'_z(a,b) &= cos^{-1}\left ( \frac{sin(lat_2)
  cos(lat_1) - cos(lat_2) sin(lat_1) * cos(lon2 - lon1)}{sin(dist(a,b))}
\right ).
\end{align*}
The difference between two azimuths $\psi(z_1, z_2)$ measures whether
$z_2$ is clockwise from $z_1$, i.e. $\psi \in [0, 180]$ or
counter-clockwise, $\psi \in [0, -180]$.
\begin{align*}
\psi(z_1, z_2) &= \begin{dcases*}
\psi'(z_1, z_2) - 360 & $\psi'(z_1, z_2) > 180$ \\
\psi'(z_1, z_2) & otherwise
\end{dcases*}
\intertext{where,}
\psi'(z_1, z_2) &= (z_2 - z_1) + 360\ \text{mod}\ 360.
\end{align*}

The difference of the detection azimuth from the theoretical
station-to-event azimuth is distributed as a Laplacian,

\[\psi(G_z(s^k_l, e^i_l), \Lambda_z^{ik}) \sim \text{Laplacian}(\ \cdot
\ ,\  \mu_z^k, \ \theta_z^k \ ) . \] 

\subsection{Detection Slowness}
The slowness at distance $\delta$ given by $I_S(\delta)$ is simply the
derivative of the travel time. In other words, slowness measures the
the time that the seismic wave takes to travel between two points very
close to the station in the direction of the event-to-station
azimuth. The theoretical slowness is given by the formula,
\[ I_S(\delta) = -.046 * \delta + 10.7\, .\]
Note that $I_S$ is always positive since $\delta \in [0, 180]$.

The detection slowness is a Laplacian centered near the theoretical
slowness,

\[ \Lambda_s^{ik} \sim \text{Laplacian}(\ \cdot \ , I_S(\Delta_{ik}) +
\mu_s^k \  , \ \theta_s^k) . \]

\subsection{Detection Amplitude}

The log of the detection amplitude has a Gaussian distribution with a mean
determined by the event magnitude and travel time.

\[\log(\Lambda_a^{ik}) \sim \text{Gaussian}(\ \cdot \ ,\ \mu^k_{a0} 
+ \mu^k_{a1} e^i_m + \mu^k_{a2} I_T(\Delta_{ik})\  ,\ \sigma_a^k \ ) . \]


\subsection{False Detections}

Each station $k$ has its own time-homogenous Poisson process generating
false detections with rate $\lambda^k_f$. In other words, if $\xi^k$ is
the set of false detections in a time interval $T$,
\begin{align*}
|\xi^k| & \sim \text{Poisson}(\ \cdot \ , \ \lambda^k_f T),
\intertext{and the detection time $\xi^k_t$ is uniformly distributed,}
\xi^k_t & \sim \text{Uniform}(\ \cdot \ , \ 0, \ T) .
\intertext{The azimuth and slowness are also uniformly distributed
  between their possible values, as follows:}
\xi^k_z & \sim \text{Uniform}(\ \cdot \ , \ 0, \ 360),  \\
\xi^k_s & \sim \text{Uniform}(\ \cdot \ , \ I_S(180), \ I_S(0)) .
\intertext{However, the log-amplitude is distributed as a Gaussian,}
\log{\xi^k_a} & \sim \text{Gaussian}(\ \cdot \ , \ \mu^k_f, \sigma^k_f) .
\end{align*}

\section{Hyperpriors and Constants}

\begin{align*}
T & = 3600 \, s \\
R & = 6371 \, km \\
\lambda_e & \sim \text{Gamma}(\ \cdot \ , \ 6.0, \frac{1}{4 \pi R^2 T}) \\
\lambda_m & = \log(10) \\
\left[ 
  \begin{array}{l}
    \mu^k_{d0} \\
    \mu^k_{d1} \\
    \mu^k_{d2} 
  \end{array} \right] & = \text{MVarGaussian} \left( \ \cdot \ ; \
\left[
  \begin{array}{l}
    -10.4 \\
    3.26 \\
    -.0499
  \end{array}
\right]
 , 
\left[
  \begin{array}{lll}
    13.43 & -2.36 & -.0122 \\
    -2.36 & .452 & .000112 \\
    -.0122 & .000112 & .000125 \\
  \end{array}
\right]
\ \right) \\
\mu_t^k & = 0 \\
\theta_t^k & \sim \text{InvGamma}(\ \cdot \ , \ 120, \ 118) \\
\mu_z^k & = 0 \\
\theta_z^k & \sim \text{InvGamma}(\ \cdot \ , \ 5.2, \ 44) \\
\mu_s^k & = 0 \\
\theta_s^k & \sim \text{InvGamma}(\ \cdot \ , \ 6.7, \ 7.5) \\
\left[ 
  \begin{array}{l}
    \mu^k_{a0} \\
    \mu^k_{a1} \\
    \mu^k_{a2} 
  \end{array} \right] & = \text{MVarGaussian} \left( \ \cdot \ ; \
\left[
  \begin{array}{l}
    -7.3 \\
    2.03 \\
    -.00196
  \end{array}
\right]
 , 
\left[
  \begin{array}{lll}
    1.23 & -.227 & -.000175 \\
    -.227 & .0461 & .0000245 \\
    -.000175 & .0000245 & .000000302 \\
  \end{array}
\right]
\ \right) \\
(\sigma^k_a)^2 & \sim \text{InvGamma}(\ \cdot \ , \  21.1, \ 12.6) \\
\lambda^k_f & \sim \text{Gamma}(\ \cdot \ , \  2.1, \ 0.0013) \\
\mu^k_f & \sim \text{Gaussian}(\ \cdot \ , \  -0.6, \ 0.6) \\
(\sigma^k_f)^2 & \sim \text{InvGamma}(\ \cdot \ , \  10.34, \ 9.49) \\
\end{align*}

\section{Stations}

\begin{tabular}{lrrr}
Abbreviation & Station Number & Longitude & Latitude \\
ASAR & 0 & 133.9 & -23.7 \\
CMAR & 1 & 98.9 & 18.5 \\
FINES & 2 & 26.1 & 61.4 \\
ILAR & 3 & -146.9 & 64.8 \\
MKAR & 4 & 82.3 & 46.8 \\
SONM & 5 & 106.4 & 47.8 \\
STKA & 6 & 141.6 & -31.9 \\
TORD & 7 & 1.7 & 13.1 \\
WRA & 8 & 134.3 & -19.9 \\
ZALV & 9 & 84.8 & 53.9 \\ 
\end{tabular}


\section{Data and Evaluation}

The distribution for this problem comes with a number of files, which
are described in the provided {\tt README.txt} file. The key point worth
noting is that we are providing a separate training and test data sets
that are generated from the same underlying physics. The file {\tt
  test.data} contains the fully labeled training data, and the file {\tt
  test.blind} is the unlabeled test data on which the predictions are to
be performed. The included file {\tt evaluate.py} will compare the
predictions to the ground truth in {\tt test.solution}.

\subsection{File Format}

Each data file consists of a number of episodes that are separated by a
blank line. Each episode has subsections for the events, detections, and
associations that took place in one episode of $T$ seconds.
The format of the blind data is identical, however it has zero events
and associations.

\begin{verbatim}
Episodes:

Events:
<longitude> <latitude> <magnitude> <time>
...
Detections:
<station number (zero-based)> <time> <azimuth> <slowness> <amplitude>
...
Assocs:
<event number (zero-based)> <detection number (zero-based)>
...

\end{verbatim}

\subsection{Evaluation}

In each episode the predicted events are matched against the ground
truth events using min-weight max-cardinality matching. Only events that
are within $W_T=50$ seconds and $W_D=5$ degrees of each other are
considered as potential matches. The weight of a matched pairs of events
$e$ and $e'$ is,
\[\frac{|e_t - e'_t|}{W_T}  + \frac{dist(e,e')}{W_D} . \]
Given a matching we can compute the precision, recall, and F-1 score in
each episode as well as over the entire data set. Additionally, we will
report the errors in magnitude estimates, distance, and time for the
matched events.

There is just one caveat here. Some of the ground truth events don't
have at least two associations, and thus can't be located
reasonably. These events have to be excluded from the matching and the
subsequent recall score.

\begin{appendices}

\section{Standard Distributions}

\subsection{Poisson}

The Poisson distribution with rate $\lambda$ has the probability density
\[ \text{Poisson}(n; \lambda) =  e ^ {-\lambda}  \, \frac{  \lambda ^ n  }{n !} \,
, \]
defined over all $n \in \mathbb{Z}_{\ge 0}$.

\subsection{Uniform}

The Uniform distribution with parameters $a, b \in \mathbb{R}$ (and $a <
b$) has the probability density
\[ \text{Uniform}(x; a, b) = \frac{1}{b - a} \, \mathbb{1}_{x>a}
\mathbb{1}_{x<b}\, ,\]
defined over all $x \in \mathbb{R}$.

\subsection{Exponential}

The Exponential distribution with rate $\lambda$ and location $l$ is
given has the probability density
\[ \text{Exponential}(x; \lambda, l) = \lambda e^{-\lambda (x - l)} \,
,\]
defined over all $x \in \mathbb{R}, x \ge l$.

\subsection{Laplacian}

The Laplacian distribution with location $\mu$ and scale $\theta$ has
probability density
\[ \text{Laplacian}(x; \mu, \theta) = \frac{1}{2 \theta} \, 
e^{-\frac{|x - \mu|}{\theta}}
\, ,\]
defined over all $x \in \mathbb{R}$.

\subsection{Gaussian}

The Gaussian distribution with mean $\mu$ and standard deviation
$\sigma$ has probability density
\[ \text{Gaussian}(x; \mu, \sigma) = \frac{1}{\sqrt{2\pi}\sigma} \,
e^{-\frac{(x - \mu)^2}{2\sigma^2}} \, ,\] 
defined over all $x \in \mathbb{R}$.

\subsection{Gamma}

The Gamma distribution with shape $\alpha$ and scale $\theta$ has probability
density 

\[\text{Gamma}(x; \alpha, \theta) = \frac{1}{\Gamma(\alpha)\,
  \theta^\alpha} \,  x^{\alpha-1}  e^{-\frac{x}{\theta}} \, , \]
defined over all $x \in \mathbb{R}_{>0}$.

\subsection{Inverse-Gamma}

The Inverse-Gamma distribution with shape $\alpha$ and scale $\theta$
has probability density

\[\text{InvGamma}(x; \alpha, \theta) = \frac{\theta^\alpha}{\Gamma(\alpha)}
 \, x^{-\alpha-1} e^{-\frac{\theta}{x}} \, , \] defined over all $x \in
 \mathbb{R}_{>0}$.

\subsection{Multi-variate Gaussian}

The Muti-variate Gaussian distribution with mean vector
$\mu \in \mathbb{R}^k$, and
covariance matrix $\Sigma  \in \mathbb{R}^{k \times k}$ has probability density

\[\text{MVarGaussian}(x; \mu, \Sigma) = \frac{1}{\sqrt{(2 \pi)^k | \Sigma
    |}}
e^{\left( -\frac{1}{2} (x - \mu)^T \Sigma^{-1} (x - \mu) \right)}
\]
defined over all $x \in \mathbb{R}^k$.
\end{appendices}

\bibliographystyle{chicagoa}
\bibliography{seismic}

\end{document}

